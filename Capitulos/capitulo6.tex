%%%%%%%%%%%%%%%%%%%%%%%%%%%%%%%%%%%%%%%%%%%%%%%%%%%%%%%%%%%%%%%%%%%%%%%%%%%%%%%
% CAPÍTULO 6

\chapter{Conclus{\~a}o}

O objetivo deste trabalho, como descrito na se{\c{c}}{\~a}o \ref{sec:objetivo}, foi desenvolver o firmware para um sistema capaz de detectar a ocorr{\^e}ncia de todos os eventos de varia{\c{c}}{\~o}es de tens{\~a}o de curta dura{\c{c}}{\~a}o descritos na tabela \ref{tab:VTCD}, salvando essa informa{\c{c}}{\~a}o, juntamente com a data e a hora da ocorr{\^e}ncia de cada evento, e disponibilizando-a online para o usu{\'a}rio atrav{\'e}s de uma rede Ethernet.  

A partir dos resultados apresentados no cap{\'i}tulo 5, {\'e} poss{\'i}vel afirmar que esse objetivo foi atingido de maneira satisfat{\'o}ria, assim como os objetivos espec{\'i}ficos tamb{\'e}m mostrados na se{\c{c}}{\~a}o \ref{sec:objetivo}. Esses resultados, obtidos atrav{\'e}s dos testes realizados com o prot{\'o}tipo e um aparelho variador de tens{\~a}o monof{\'a}sico, mostram que o sistema {\'e} capaz de detectar e diferenciar eventos de interrup{\c{c}}{\~a}o, afundamento e eleva{\c{c}}{\~a}o de tens{\~a}o, assim como classific{\'a}-los de acordo com sua dura{\c{c}}{\~a}o.

A comunica{\c{c}}{\~a}o entre o microcontrolador e o circuito integrado ADE7758, realizada atrav{\'e}s do protocolo de comunica{\c{c}}{\~a}o serial SPI descrito na se{\c{c}}{\~a}o \label{sec:spi}, funcionou corretamente. A configura{\c{c}}{\~a}o das interrup{\c{c}}{\~o}es para cada uma das fases da rede e a leitura dos valores de tens{\~a}o medidos pelo circuito integrado tiveram seu funcionamento comprovado nos testes do cap{\'i}tulo 5.

Os testes tamb{\'e}m mostraram que o sistema teve um bom desempenho para todos os tipos de eventos de VTCDs, sendo capaz de identificar interrup{\c{c}}{\~o}es, afundamentos e eleva{\c{c}}{\~o}es. A classifica{\c{c}}{\~a}o desses eventos quanto a sua dura{\c{c}}{\~a}o tamb{\'e}m foi feita de maneira satisfat{\'o}ria, mesmo nos casos em que a dura{\c{c}}{\~a}o foi muito pr{\'o}xima dos limiares.

A capacidade do sistema de salvar data e hora de cada ocorr{\^e}ncia pode ser comprovada com a utiliza{\c{c}}{\~a}o do software \textit{Hercules}. As mensagens CDT e VDT, descritas na se{\c{c}}{\~a}o \ref{sec:mess}, mostram isso. As respostas do sistema {\`a}s mensagens FA, FB e FC mostram tamb{\'e}m o correto funcionamento do buffer rotativo. 

Os testes realizados com o prot{\'o}tipo e o software \textit{Hercules} tamb{\'e}m demonstraram o correto funcionamento da rede Ethernet em conjunto com a arquitetura TCP/IP. As respostas do dispositivo, configurado como servidor, {\`a}s requisi{\c{c}}{\~o}es do terminal TCP Client, demonstram isso.

Apesar de o sistema ter, no geral, funcionado de maneira correta, foram identificados alguns problemas que devem ser solucionados para trabalhos futuros. Como descrito na se{\c{c}}{\~a}o \ref{subsec:rms}, o sistema aguarda 100 ms entre as opera{\c{c}}{\~o}es de leitura dos registradores de tens{\~a}o do circuito integrado ADE7758. Isso significa que o valor eficaz da tens{\~a}o das tr{\^e}s fases da rede s{\'o} {\'e} atualizado 10 vezes por segundo, uma taxa menor do que a necess{\'a}ria para uma rede de 60 Hz. Como consequ{\^e}ncia, essa diferen{\c{c}}a pode ocasionar erros na classifica{\c{c}}{\~a}o da dura{\c{c}}{\~a}o dos eventos de VTCD.

Outro ponto {\'e} referente a alimenta{\c{c}}{\~a}o da placa de medi{\c{c}}{\~a}o trif{\'a}sica, que {\'e} proveniente da mesma rede que est{\'a} sendo monitorada. Desta maneira, os eventos de VTCD podem ocasionar problemas no funcionamento do dispositivo, como um desligamento durante um evento de interrup{\c{c}}{\~a}o e a perda das informa{\c{c}}{\~o}es salvas no buffer rotativo, j{\'a} que este utiliza a mem{\'o}ria RAM do microcontrolador.

Ap{\'o}s a conclus{\~a}o deste trabalho, {\'e} poss{\'i}vel destacar diversos pontos que podem ser explorados em trabalhos futuros:

\begin{itemize}
\item [-] Um estudo aprofundado da classifica{\c{c}}{\~a}o de severidade de VTCDs, definida pela Aneel. Essa severidade {\'e} calculada pela frequ{\^e}ncia de ocorr{\^e}ncia de eventos de VTCD em um determinado per{\'i}odo de tempo, levando-se em conta tamb{\'e}m os crit{\'e}rios de agrega{\c{c}}{\~a}o de eventos consecutivos por faixas de amplitude e dura{\c{c}}{\~a}o. A partir desse levantamento, {\'e} poss{\'i}vel fazer uma correla{\c{c}}{\~a}o entre a sensibilidade de cargas conectadas {\`a} rede el{\'e}trica e a import{\^a}ncia de cada evento de VTCD.
\item [-] Expandir o funcionamento do sistema de detec{\c{c}}{\~a}o para outros tipos de problemas de qualidade de energia, como harm{\^o}nicos, varia{\c{c}}{\~o}es de tens{\~a}o de longa dura{\c{c}}{\~a}o, flutua{\c{c}}{\~o}es na rede, etc. 
\item [-] Conectar o sistema a um banco de dados, permitindo assim salvar as informa{\c{c}}{\~o}es relativas ao eventos de VTCDs mesmo ap{\'o}s o desligamento do dispositivo.
\item [-] Realizar testes individuais mais complexos em cada uma das fases da rede el{\'e}trica, permitindo uma an{\'a}lise mais detalhada da resposta do sistema para diferentes eventos de VTCDs ocorrendo simultaneamente em fases diferentes.
\item [-] Estudar uma configura{\c{c}}{\~a}o diferente para o sistema, com o microcontrolador programado como um cliente na rede TCP/IP, recebendo os eventos ap{\'o}s sua ocorr{\^e}ncia (e n{\~a}o requisitando a informa{\c{c}}{\~a}o).
\item [-] Estudar a utiliza{\c{c}}{\~a}o de um microcontrolador de menor custo para o prot{\'o}tipo, tendo em vista que o uso de um modelo da fam{\'i}lia PIC32MX foi justificado por uma necessidade elevada de processamento e mem{\'o}ria para os c{\'a}lculos fasoriais em \citeonline{fonsecadesenvolvimento}, mas acabou se mostrando acima da capacidade necess{\'a}ria para este trabalho.
\end{itemize}