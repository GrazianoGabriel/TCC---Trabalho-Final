%%%%%%%%%%%%%%%%%%%%%%%%%%%%%%%%%%%%%%%%%%%%%%%%%%%%%%%%%%%%%%%%%%%%%%%%%%%%%%%
% CAPÍTULO 6

\chapter{Conclus{\~a}o}

O objetivo deste trabalho, como descrito na se{\c{c}}{\~a}o \ref{sec:objetivo}, era desenvolver um sistema capaz de detectar a ocorr{\^e}ncia de todos os eventos de varia{\c{c}}{\~o}es de tens{\~a}o de curta dura{\c{c}}{\~a}o descritos na tabela \ref{tab:VTCD}, salvando essa informa{\c{c}}{\~a}o, juntamente com a data e a hora da ocor{\^e}ncia de cada evento, e disponibilizando-a online para o usu{\'a}rio atrav{\'e}s de uma rede Ethernet.  

A partir dos resultados apresentados no cap{\'i}tulo 5, {\'e} poss{\'e}vel afirmar que esse objetivo foi atingido de maneira satisfat{\'o}ria, assim como os objetivos espec{\'i}ficos tamb{\'e}m mostrados na se{\c{c}}{\~a}o \ref{sec:objetivo}. Esses resultados, obtidos atrav{\'e}s dos testes realizados com o prot{\'o}tipo e um aparelho variador de tens{\~a}o monof{\'a}sico, mostram que o sistema {\'e} capaz de detectar e diferenciar eventos de interrup{\c{c}}{\~a}o, afundamento e eleva{\c{c}}{\~a}o de tens{\~a}o, assim como classific{\'a}-los de acordo com sua dura{\c{c}}{\~a}o.